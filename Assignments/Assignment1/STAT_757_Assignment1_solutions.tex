\documentclass[]{article}
\usepackage{lmodern}
\usepackage{amssymb,amsmath}
\usepackage{ifxetex,ifluatex}
\usepackage{fixltx2e} % provides \textsubscript
\ifnum 0\ifxetex 1\fi\ifluatex 1\fi=0 % if pdftex
  \usepackage[T1]{fontenc}
  \usepackage[utf8]{inputenc}
\else % if luatex or xelatex
  \ifxetex
    \usepackage{mathspec}
  \else
    \usepackage{fontspec}
  \fi
  \defaultfontfeatures{Ligatures=TeX,Scale=MatchLowercase}
\fi
% use upquote if available, for straight quotes in verbatim environments
\IfFileExists{upquote.sty}{\usepackage{upquote}}{}
% use microtype if available
\IfFileExists{microtype.sty}{%
\usepackage{microtype}
\UseMicrotypeSet[protrusion]{basicmath} % disable protrusion for tt fonts
}{}
\usepackage[margin=1in]{geometry}
\usepackage{hyperref}
\hypersetup{unicode=true,
            pdftitle={STAT 757 - Assignment 1},
            pdfauthor={Jane Doe},
            pdfborder={0 0 0},
            breaklinks=true}
\urlstyle{same}  % don't use monospace font for urls
\usepackage{color}
\usepackage{fancyvrb}
\newcommand{\VerbBar}{|}
\newcommand{\VERB}{\Verb[commandchars=\\\{\}]}
\DefineVerbatimEnvironment{Highlighting}{Verbatim}{commandchars=\\\{\}}
% Add ',fontsize=\small' for more characters per line
\usepackage{framed}
\definecolor{shadecolor}{RGB}{248,248,248}
\newenvironment{Shaded}{\begin{snugshade}}{\end{snugshade}}
\newcommand{\AlertTok}[1]{\textcolor[rgb]{0.94,0.16,0.16}{#1}}
\newcommand{\AnnotationTok}[1]{\textcolor[rgb]{0.56,0.35,0.01}{\textbf{\textit{#1}}}}
\newcommand{\AttributeTok}[1]{\textcolor[rgb]{0.77,0.63,0.00}{#1}}
\newcommand{\BaseNTok}[1]{\textcolor[rgb]{0.00,0.00,0.81}{#1}}
\newcommand{\BuiltInTok}[1]{#1}
\newcommand{\CharTok}[1]{\textcolor[rgb]{0.31,0.60,0.02}{#1}}
\newcommand{\CommentTok}[1]{\textcolor[rgb]{0.56,0.35,0.01}{\textit{#1}}}
\newcommand{\CommentVarTok}[1]{\textcolor[rgb]{0.56,0.35,0.01}{\textbf{\textit{#1}}}}
\newcommand{\ConstantTok}[1]{\textcolor[rgb]{0.00,0.00,0.00}{#1}}
\newcommand{\ControlFlowTok}[1]{\textcolor[rgb]{0.13,0.29,0.53}{\textbf{#1}}}
\newcommand{\DataTypeTok}[1]{\textcolor[rgb]{0.13,0.29,0.53}{#1}}
\newcommand{\DecValTok}[1]{\textcolor[rgb]{0.00,0.00,0.81}{#1}}
\newcommand{\DocumentationTok}[1]{\textcolor[rgb]{0.56,0.35,0.01}{\textbf{\textit{#1}}}}
\newcommand{\ErrorTok}[1]{\textcolor[rgb]{0.64,0.00,0.00}{\textbf{#1}}}
\newcommand{\ExtensionTok}[1]{#1}
\newcommand{\FloatTok}[1]{\textcolor[rgb]{0.00,0.00,0.81}{#1}}
\newcommand{\FunctionTok}[1]{\textcolor[rgb]{0.00,0.00,0.00}{#1}}
\newcommand{\ImportTok}[1]{#1}
\newcommand{\InformationTok}[1]{\textcolor[rgb]{0.56,0.35,0.01}{\textbf{\textit{#1}}}}
\newcommand{\KeywordTok}[1]{\textcolor[rgb]{0.13,0.29,0.53}{\textbf{#1}}}
\newcommand{\NormalTok}[1]{#1}
\newcommand{\OperatorTok}[1]{\textcolor[rgb]{0.81,0.36,0.00}{\textbf{#1}}}
\newcommand{\OtherTok}[1]{\textcolor[rgb]{0.56,0.35,0.01}{#1}}
\newcommand{\PreprocessorTok}[1]{\textcolor[rgb]{0.56,0.35,0.01}{\textit{#1}}}
\newcommand{\RegionMarkerTok}[1]{#1}
\newcommand{\SpecialCharTok}[1]{\textcolor[rgb]{0.00,0.00,0.00}{#1}}
\newcommand{\SpecialStringTok}[1]{\textcolor[rgb]{0.31,0.60,0.02}{#1}}
\newcommand{\StringTok}[1]{\textcolor[rgb]{0.31,0.60,0.02}{#1}}
\newcommand{\VariableTok}[1]{\textcolor[rgb]{0.00,0.00,0.00}{#1}}
\newcommand{\VerbatimStringTok}[1]{\textcolor[rgb]{0.31,0.60,0.02}{#1}}
\newcommand{\WarningTok}[1]{\textcolor[rgb]{0.56,0.35,0.01}{\textbf{\textit{#1}}}}
\usepackage{graphicx,grffile}
\makeatletter
\def\maxwidth{\ifdim\Gin@nat@width>\linewidth\linewidth\else\Gin@nat@width\fi}
\def\maxheight{\ifdim\Gin@nat@height>\textheight\textheight\else\Gin@nat@height\fi}
\makeatother
% Scale images if necessary, so that they will not overflow the page
% margins by default, and it is still possible to overwrite the defaults
% using explicit options in \includegraphics[width, height, ...]{}
\setkeys{Gin}{width=\maxwidth,height=\maxheight,keepaspectratio}
\IfFileExists{parskip.sty}{%
\usepackage{parskip}
}{% else
\setlength{\parindent}{0pt}
\setlength{\parskip}{6pt plus 2pt minus 1pt}
}
\setlength{\emergencystretch}{3em}  % prevent overfull lines
\providecommand{\tightlist}{%
  \setlength{\itemsep}{0pt}\setlength{\parskip}{0pt}}
\setcounter{secnumdepth}{0}
% Redefines (sub)paragraphs to behave more like sections
\ifx\paragraph\undefined\else
\let\oldparagraph\paragraph
\renewcommand{\paragraph}[1]{\oldparagraph{#1}\mbox{}}
\fi
\ifx\subparagraph\undefined\else
\let\oldsubparagraph\subparagraph
\renewcommand{\subparagraph}[1]{\oldsubparagraph{#1}\mbox{}}
\fi

%%% Use protect on footnotes to avoid problems with footnotes in titles
\let\rmarkdownfootnote\footnote%
\def\footnote{\protect\rmarkdownfootnote}

%%% Change title format to be more compact
\usepackage{titling}

% Create subtitle command for use in maketitle
\newcommand{\subtitle}[1]{
  \posttitle{
    \begin{center}\large#1\end{center}
    }
}

\setlength{\droptitle}{-2em}
  \title{STAT 757 - Assignment 1}
  \pretitle{\vspace{\droptitle}\centering\huge}
  \posttitle{\par}
  \author{Jane Doe}
  \preauthor{\centering\large\emph}
  \postauthor{\par}
  \predate{\centering\large\emph}
  \postdate{\par}
  \date{February 2, 2018}


\begin{document}
\maketitle

\hypertarget{i.-datacamp-introduction-to-r-30-points}{%
\subsection{i. DataCamp: Introduction to R {[}30
points{]}}\label{i.-datacamp-introduction-to-r-30-points}}

Please complete the course an Introduction to R. You should have
received an email with an invitation link. Please email me if you did
not. If you already know R, please talk to me in class and follow up
with an email to opt out.

Completed. Records are in DataCamp.

\hypertarget{ii.-instructions-for-the-rest-of-this-assignment}{%
\subsection{ii. Instructions for the rest of this
assignment}\label{ii.-instructions-for-the-rest-of-this-assignment}}

The purpose of this portion of the assignment is to get a little
experience making R Markdown documents as a way of nicely formatting
output from R code while exploring the datasets from Sheather Ch.1 and
learning to generate realizations of random variables (aka ``fake
data''). Modify this RMarkdown file (STAT\_757\_Assignment1.Rmd) and
compile your document as a PDF (or Word document if you're having LaTeX
issuse) and naming it according to the format
SURNAME-FIRSTNAME-Assignment1.pdf, and emailing that PDF to the
instructor by the due date listed above.

\hypertarget{reproduce-the-plots-from-sheather-ch.1-40-points}{%
\subsection{2. Reproduce the plots from Sheather Ch.1 {[}40
points{]}}\label{reproduce-the-plots-from-sheather-ch.1-40-points}}

Modify this file so that it reproduces all the output from the R script
located at
\href{http://www.stat.\%20tamu.edu/~sheather/book/docs/rcode/Chapter1.R}{http://www.stat.
tamu.edu/\textasciitilde{}sheather/book/docs/rcode/Chapter1.R}. I've
done the plots for the first dataset for you below. Remember that you
will need to download each of the four data sets from
\url{http://www.stat.tamu.edu/~sheather/book/data_sets.php}, and set
your working directory (under the ``Session'' menu in Rstudio)
appropriately. (And yes, this really is as easy as copying the blocks of
R code for each dataset into this document into the appropriate places!)
Need help? First, see \url{http://rmarkdown.rstudio.com}. Especially the
resources under Learning More
(\url{http://rmarkdown.rstudio.com/\#learning-more}).

Below are the plots that appear in Chapter 1 of the textbook. They were
created from the R script
\url{http://www.stat.tamu.edu/~sheather/book/docs/rcode/Chapter1.R} and
the data files at
\url{http://www.stat.tamu.edu/~sheather/book/data_sets.php}.

\hypertarget{assessing-the-ability-of-nfl-kickers}{%
\subsubsection{Assessing the ability of NFL
Kickers}\label{assessing-the-ability-of-nfl-kickers}}

\begin{Shaded}
\begin{Highlighting}[]
\NormalTok{kicker <-}\StringTok{ }\KeywordTok{read.csv}\NormalTok{(}\StringTok{"~/OneDrive - University of Nevada, Reno/Teaching/STAT_757/Sheather_data/Data/FieldGoals2003to2006.csv"}\NormalTok{,}\DataTypeTok{header=}\NormalTok{T) ## adjust this as needed.}
\NormalTok{## Sorry this line is too long, the data are labeled 'FieldGoals2003to2006.csv'}

\KeywordTok{attach}\NormalTok{(kicker) ## THIS IS NOT USUALLY RECOMMENDED, ASK ME IN CLASS WHY NOT.}

\CommentTok{#Figure 1.1 on page 2}
\KeywordTok{plot}\NormalTok{(kicker}\OperatorTok{$}\NormalTok{FGtM1,kicker}\OperatorTok{$}\NormalTok{FGt,}
\DataTypeTok{main=}\StringTok{"Unadjusted Correlation = -0.139"}\NormalTok{,}
\DataTypeTok{xlab=}\StringTok{"Field Goal Percentage in Year t-1"}\NormalTok{,}\DataTypeTok{ylab=}\StringTok{"Field Goal Percentage in Year t"}\NormalTok{)}
\end{Highlighting}
\end{Shaded}

\includegraphics{STAT_757_Assignment1_solutions_files/figure-latex/unnamed-chunk-1-1.pdf}

\begin{Shaded}
\begin{Highlighting}[]
\CommentTok{#p-values on page 3}
\NormalTok{fit}\FloatTok{.1}\NormalTok{ <-}\StringTok{ }\KeywordTok{lm}\NormalTok{(FGt}\OperatorTok{~}\NormalTok{FGtM1 }\OperatorTok{+}\NormalTok{Name }\OperatorTok{+}\NormalTok{FGtM1}\OperatorTok{:}\NormalTok{Name,}\DataTypeTok{data=}\NormalTok{kicker)}
\KeywordTok{anova}\NormalTok{(fit}\FloatTok{.1}\NormalTok{)}
\end{Highlighting}
\end{Shaded}

\begin{verbatim}
## Analysis of Variance Table
## 
## Response: FGt
##            Df Sum Sq Mean Sq F value Pr(>F)
## FGtM1       1     87    87.2    1.90 0.1760
## Name       18   2252   125.1    2.73 0.0046
## FGtM1:Name 18    418    23.2    0.51 0.9386
## Residuals  38   1743    45.9
\end{verbatim}

\begin{Shaded}
\begin{Highlighting}[]
\CommentTok{#slope and interecepts of lines in Figure 1.2 on page 3}
\NormalTok{fit}\FloatTok{.2}\NormalTok{ <-}\StringTok{ }\KeywordTok{lm}\NormalTok{(FGt }\OperatorTok{~}\StringTok{ }\NormalTok{Name }\OperatorTok{+}\StringTok{ }\NormalTok{FGtM1,}\DataTypeTok{data=}\NormalTok{kicker)}
\NormalTok{fit}\FloatTok{.2}
\end{Highlighting}
\end{Shaded}

\begin{verbatim}
## 
## Call:
## lm(formula = FGt ~ Name + FGtM1, data = kicker)
## 
## Coefficients:
##              (Intercept)           NameDavid Akers  
##                  126.687                    -4.646  
##           NameJason Elam          NameJason Hanson  
##                   -3.017                     2.117  
##            NameJay Feely             NameJeff Reed  
##                  -10.374                    -8.296  
##         NameJeff Wilkins           NameJohn Carney  
##                    2.310                    -5.977  
##            NameJohn Hall            NameKris Brown  
##                   -8.486                   -13.360  
##          NameMatt Stover       NameMike Vanderjagt  
##                    8.736                     4.896  
##         NameNeil Rackers           NameOlindo Mare  
##                   -6.620                   -13.036  
##          NamePhil Dawson          NameRian Lindell  
##                    3.552                    -4.867  
##        NameRyan Longwell  NameSebastian Janikowski  
##                   -2.231                    -3.976  
##        NameShayne Graham                     FGtM1  
##                    2.135                    -0.504
\end{verbatim}

\begin{Shaded}
\begin{Highlighting}[]
\CommentTok{#Figure 1.2 on page 3}
\KeywordTok{plot}\NormalTok{(kicker}\OperatorTok{$}\NormalTok{FGtM1,kicker}\OperatorTok{$}\NormalTok{FGt,}
\DataTypeTok{main=}\StringTok{"Slope of each line = -0.504"}\NormalTok{,}
\DataTypeTok{xlab=}\StringTok{"Field Goal Percentage in Year t-1"}\NormalTok{,}
\DataTypeTok{ylab=}\StringTok{"Field Goal Percentage in Year t"}\NormalTok{)}
\NormalTok{tt <-}\StringTok{ }\KeywordTok{seq}\NormalTok{(}\DecValTok{60}\NormalTok{,}\DecValTok{100}\NormalTok{,}\DataTypeTok{length=}\DecValTok{1001}\NormalTok{)}
\NormalTok{slope.piece <-}\StringTok{ }\KeywordTok{summary}\NormalTok{(fit}\FloatTok{.2}\NormalTok{)}\OperatorTok{$}\NormalTok{coef[}\DecValTok{20}\NormalTok{]}\OperatorTok{*}\NormalTok{tt}
\KeywordTok{lines}\NormalTok{(tt,}\KeywordTok{summary}\NormalTok{(fit}\FloatTok{.2}\NormalTok{)}\OperatorTok{$}\NormalTok{coef[}\DecValTok{1}\NormalTok{]}\OperatorTok{+}\NormalTok{slope.piece,}\DataTypeTok{lty=}\DecValTok{2}\NormalTok{)}
\ControlFlowTok{for}\NormalTok{ (i }\ControlFlowTok{in} \DecValTok{2}\OperatorTok{:}\DecValTok{19}\NormalTok{)}
\NormalTok{\{}\KeywordTok{lines}\NormalTok{(tt,}\KeywordTok{summary}\NormalTok{(fit}\FloatTok{.2}\NormalTok{)}\OperatorTok{$}\NormalTok{coef[}\DecValTok{1}\NormalTok{]}\OperatorTok{+}\KeywordTok{summary}\NormalTok{(fit}\FloatTok{.2}\NormalTok{)}\OperatorTok{$}\NormalTok{coef[i]}\OperatorTok{+}\NormalTok{slope.piece,}\DataTypeTok{lty=}\DecValTok{2}\NormalTok{)\}}
\end{Highlighting}
\end{Shaded}

\includegraphics{STAT_757_Assignment1_solutions_files/figure-latex/unnamed-chunk-1-2.pdf}

\begin{Shaded}
\begin{Highlighting}[]
\KeywordTok{detach}\NormalTok{(kicker)}
\end{Highlighting}
\end{Shaded}

\#\#\#Newspaper circulation

\begin{Shaded}
\begin{Highlighting}[]
\NormalTok{circulation <-}\StringTok{ }\KeywordTok{read.table}\NormalTok{(}\StringTok{"~/OneDrive - University of Nevada, Reno/Teaching/STAT_757/Sheather_data/Data/circulation.txt"}\NormalTok{, }\DataTypeTok{header=}\OtherTok{TRUE}\NormalTok{, }\DataTypeTok{sep=}\StringTok{"}\CharTok{\textbackslash{}t}\StringTok{"}\NormalTok{)}
\KeywordTok{attach}\NormalTok{(circulation)}

\CommentTok{#Figure 1.3 on page 5}
\KeywordTok{plot}\NormalTok{(Weekday,Sunday,}\DataTypeTok{xlab=}\StringTok{"Weekday Circulation"}\NormalTok{,}\DataTypeTok{ylab=}\StringTok{"Sunday Circulation"}\NormalTok{,}
\DataTypeTok{pch=}\NormalTok{Tabloid.with.a.Serious.Competitor}\OperatorTok{+}\DecValTok{1}\NormalTok{,}\DataTypeTok{col=}\NormalTok{Tabloid.with.a.Serious.Competitor}\OperatorTok{+}\DecValTok{1}\NormalTok{)}
\KeywordTok{legend}\NormalTok{(}\DecValTok{110000}\NormalTok{, }\DecValTok{1600000}\NormalTok{,}\DataTypeTok{legend=}\KeywordTok{c}\NormalTok{(}\StringTok{"0"}\NormalTok{,}\StringTok{"1"}\NormalTok{),}
\DataTypeTok{pch=}\DecValTok{1}\OperatorTok{:}\DecValTok{2}\NormalTok{,}\DataTypeTok{col=}\DecValTok{1}\OperatorTok{:}\DecValTok{2}\NormalTok{,}\DataTypeTok{title=}\StringTok{"Tabloid dummy variable"}\NormalTok{)}
\end{Highlighting}
\end{Shaded}

\includegraphics{STAT_757_Assignment1_solutions_files/figure-latex/unnamed-chunk-2-1.pdf}

\begin{Shaded}
\begin{Highlighting}[]
\CommentTok{#Figure 1.4 on page 5}
\KeywordTok{plot}\NormalTok{(}\KeywordTok{log}\NormalTok{(Weekday),}\KeywordTok{log}\NormalTok{(Sunday),}\DataTypeTok{xlab=}\StringTok{"log(Weekday Circulation)"}\NormalTok{,}\DataTypeTok{ylab=}\StringTok{"log(Sunday Circulation)"}\NormalTok{,}
\DataTypeTok{pch=}\NormalTok{Tabloid.with.a.Serious.Competitor}\OperatorTok{+}\DecValTok{1}\NormalTok{,}
\DataTypeTok{col=}\NormalTok{Tabloid.with.a.Serious.Competitor}\OperatorTok{+}\DecValTok{1}\NormalTok{)}
\KeywordTok{legend}\NormalTok{(}\FloatTok{11.6}\NormalTok{, }\FloatTok{14.1}\NormalTok{,}\DataTypeTok{legend=}\KeywordTok{c}\NormalTok{(}\StringTok{"0"}\NormalTok{,}\StringTok{"1"}\NormalTok{),}\DataTypeTok{pch=}\DecValTok{1}\OperatorTok{:}\DecValTok{2}\NormalTok{,}\DataTypeTok{col=}\DecValTok{1}\OperatorTok{:}\DecValTok{2}\NormalTok{,}
\DataTypeTok{title=}\StringTok{"Tabloid dummy variable"}\NormalTok{)}
\end{Highlighting}
\end{Shaded}

\includegraphics{STAT_757_Assignment1_solutions_files/figure-latex/unnamed-chunk-2-2.pdf}

\begin{Shaded}
\begin{Highlighting}[]
\KeywordTok{detach}\NormalTok{(circulation)}
\end{Highlighting}
\end{Shaded}

\hypertarget{menu-pricing-in-a-new-italian-restaurant-in-nyc}{%
\subsubsection{Menu pricing in a new Italian restaurant in
NYC}\label{menu-pricing-in-a-new-italian-restaurant-in-nyc}}

\begin{Shaded}
\begin{Highlighting}[]
\NormalTok{nyc <-}\StringTok{ }\KeywordTok{read.csv}\NormalTok{(}\StringTok{"~/OneDrive - University of Nevada, Reno/Teaching/STAT_757/Sheather_data/Data/nyc.csv"}\NormalTok{,}\DataTypeTok{header=}\OtherTok{TRUE}\NormalTok{)}
\KeywordTok{attach}\NormalTok{(nyc)}

\CommentTok{#Figure 1.5 on page 7}
\KeywordTok{pairs}\NormalTok{(Price}\OperatorTok{~}\NormalTok{Food}\OperatorTok{+}\NormalTok{Decor}\OperatorTok{+}\NormalTok{Service,}\DataTypeTok{data=}\NormalTok{nyc,}\DataTypeTok{gap=}\FloatTok{0.4}\NormalTok{,}
\DataTypeTok{cex.labels=}\FloatTok{1.5}\NormalTok{)}
\end{Highlighting}
\end{Shaded}

\includegraphics{STAT_757_Assignment1_solutions_files/figure-latex/unnamed-chunk-3-1.pdf}

\begin{Shaded}
\begin{Highlighting}[]
\CommentTok{#Figure 1.6 on page 10}
\KeywordTok{boxplot}\NormalTok{(Price}\OperatorTok{~}\NormalTok{East,}\DataTypeTok{ylab=}\StringTok{"Price"}\NormalTok{,}
\DataTypeTok{xlab=}\StringTok{"East (1 = East of Fifth Avenue)"}\NormalTok{)}
\end{Highlighting}
\end{Shaded}

\includegraphics{STAT_757_Assignment1_solutions_files/figure-latex/unnamed-chunk-3-2.pdf}

\begin{Shaded}
\begin{Highlighting}[]
\KeywordTok{detach}\NormalTok{(nyc)}
\end{Highlighting}
\end{Shaded}

\hypertarget{effect-of-wine-critics-ratings-on-prices-of-bourdeax-wines}{%
\subsubsection{Effect of wine critics' ratings on prices of Bourdeax
wines}\label{effect-of-wine-critics-ratings-on-prices-of-bourdeax-wines}}

\begin{Shaded}
\begin{Highlighting}[]
\NormalTok{Bordeaux <-}\StringTok{ }\KeywordTok{read.csv}\NormalTok{(}\StringTok{"~/OneDrive - University of Nevada, Reno/Teaching/STAT_757/Sheather_data/Data/Bordeaux.csv"}\NormalTok{, }\DataTypeTok{header=}\OtherTok{TRUE}\NormalTok{)}
\KeywordTok{attach}\NormalTok{(Bordeaux)}

\CommentTok{#Figure 1.7 on page 10}
\KeywordTok{pairs}\NormalTok{(Price}\OperatorTok{~}\NormalTok{ParkerPoints}\OperatorTok{+}\NormalTok{CoatesPoints,}\DataTypeTok{data=}\NormalTok{Bordeaux,}\DataTypeTok{gap=}\FloatTok{0.4}\NormalTok{,}\DataTypeTok{cex.labels=}\FloatTok{1.5}\NormalTok{)}
\end{Highlighting}
\end{Shaded}

\includegraphics{STAT_757_Assignment1_solutions_files/figure-latex/unnamed-chunk-4-1.pdf}

\begin{Shaded}
\begin{Highlighting}[]
\CommentTok{#Figure 1.8 on page 11}
\KeywordTok{par}\NormalTok{(}\DataTypeTok{mfrow=}\KeywordTok{c}\NormalTok{(}\DecValTok{2}\NormalTok{,}\DecValTok{3}\NormalTok{))}
\KeywordTok{boxplot}\NormalTok{(Price}\OperatorTok{~}\NormalTok{P95andAbove,}\DataTypeTok{ylab=}\StringTok{"Price"}\NormalTok{,}\DataTypeTok{xlab=}\StringTok{"P95andAbove"}\NormalTok{)}
\KeywordTok{boxplot}\NormalTok{(Price}\OperatorTok{~}\NormalTok{FirstGrowth,}\DataTypeTok{ylab=}\StringTok{"Price"}\NormalTok{,}\DataTypeTok{xlab=}\StringTok{"First Growth"}\NormalTok{)}
\KeywordTok{boxplot}\NormalTok{(Price}\OperatorTok{~}\NormalTok{CultWine,}\DataTypeTok{ylab=}\StringTok{"Price"}\NormalTok{,}\DataTypeTok{xlab=}\StringTok{"Cult Wine"}\NormalTok{)}
\KeywordTok{boxplot}\NormalTok{(Price}\OperatorTok{~}\NormalTok{Pomerol,}\DataTypeTok{ylab=}\StringTok{"Price"}\NormalTok{,}\DataTypeTok{xlab=}\StringTok{"Pomerol"}\NormalTok{)}
\KeywordTok{boxplot}\NormalTok{(Price}\OperatorTok{~}\NormalTok{VintageSuperstar,}\DataTypeTok{ylab=}\StringTok{"Price"}\NormalTok{,}\DataTypeTok{xlab=}\StringTok{"Vintage Superstar"}\NormalTok{)}

\CommentTok{#Figure 1.9 on page 12}
\KeywordTok{par}\NormalTok{(}\DataTypeTok{mfrow=}\KeywordTok{c}\NormalTok{(}\DecValTok{1}\NormalTok{,}\DecValTok{1}\NormalTok{))}
\end{Highlighting}
\end{Shaded}

\includegraphics{STAT_757_Assignment1_solutions_files/figure-latex/unnamed-chunk-4-2.pdf}

\begin{Shaded}
\begin{Highlighting}[]
\KeywordTok{pairs}\NormalTok{(}\KeywordTok{log}\NormalTok{(Price)}\OperatorTok{~}\KeywordTok{log}\NormalTok{(ParkerPoints)}\OperatorTok{+}\KeywordTok{log}\NormalTok{(CoatesPoints),}\DataTypeTok{data=}\NormalTok{Bordeaux,}\DataTypeTok{gap=}\FloatTok{0.4}\NormalTok{,}\DataTypeTok{cex.labels=}\FloatTok{1.5}\NormalTok{)}
\end{Highlighting}
\end{Shaded}

\includegraphics{STAT_757_Assignment1_solutions_files/figure-latex/unnamed-chunk-4-3.pdf}

\begin{Shaded}
\begin{Highlighting}[]
\CommentTok{#Figure 1.10 on page 13}
\KeywordTok{par}\NormalTok{(}\DataTypeTok{mfrow=}\KeywordTok{c}\NormalTok{(}\DecValTok{2}\NormalTok{,}\DecValTok{3}\NormalTok{))}
\KeywordTok{boxplot}\NormalTok{(}\KeywordTok{log}\NormalTok{(Price)}\OperatorTok{~}\NormalTok{P95andAbove,}\DataTypeTok{ylab=}\StringTok{"log(Price)"}\NormalTok{,}
\DataTypeTok{xlab=}\StringTok{"P95andAbove"}\NormalTok{)}
\KeywordTok{boxplot}\NormalTok{(}\KeywordTok{log}\NormalTok{(Price)}\OperatorTok{~}\NormalTok{FirstGrowth,}\DataTypeTok{ylab=}\StringTok{"log(Price)"}\NormalTok{,}
\DataTypeTok{xlab=}\StringTok{"First Growth"}\NormalTok{)}
\KeywordTok{boxplot}\NormalTok{(}\KeywordTok{log}\NormalTok{(Price)}\OperatorTok{~}\NormalTok{CultWine,}\DataTypeTok{ylab=}\StringTok{"log(Price)"}\NormalTok{,}
\DataTypeTok{xlab=}\StringTok{"Cult Wine"}\NormalTok{)}
\KeywordTok{boxplot}\NormalTok{(}\KeywordTok{log}\NormalTok{(Price)}\OperatorTok{~}\NormalTok{Pomerol,}\DataTypeTok{ylab=}\StringTok{"log(Price)"}\NormalTok{,}
\DataTypeTok{xlab=}\StringTok{"Pomerol"}\NormalTok{)}
\KeywordTok{boxplot}\NormalTok{(}\KeywordTok{log}\NormalTok{(Price)}\OperatorTok{~}\NormalTok{VintageSuperstar,}\DataTypeTok{ylab=}\StringTok{"log(Price)"}\NormalTok{,}
\DataTypeTok{xlab=}\StringTok{"Vintage Superstar"}\NormalTok{)}

\KeywordTok{detach}\NormalTok{(Bordeaux)}
\end{Highlighting}
\end{Shaded}

\includegraphics{STAT_757_Assignment1_solutions_files/figure-latex/unnamed-chunk-4-4.pdf}

\hypertarget{generating-fake-data-30-points}{%
\subsection{3. Generating fake data {[}30
points{]}}\label{generating-fake-data-30-points}}

\hypertarget{generate-100-random-variates-from-a-normal-distribution-with-mean-0-and-standard-deviation-of-100.-summarize-and-plot-the-data.-set-a-seed-to-make-it-reproducible.}{%
\subsubsection{3.1 Generate 100 random variates from a normal
distribution with mean 0 and standard deviation of 100. Summarize and
plot the data. (Set a seed to make it
reproducible).}\label{generate-100-random-variates-from-a-normal-distribution-with-mean-0-and-standard-deviation-of-100.-summarize-and-plot-the-data.-set-a-seed-to-make-it-reproducible.}}

\begin{Shaded}
\begin{Highlighting}[]
\KeywordTok{set.seed}\NormalTok{(}\DecValTok{100}\NormalTok{)}
\NormalTok{sample1<-}\KeywordTok{rnorm}\NormalTok{(}\DecValTok{100}\NormalTok{,}\DecValTok{0}\NormalTok{,}\DecValTok{100}\NormalTok{)}
\KeywordTok{hist}\NormalTok{(sample1,}\DataTypeTok{xlab=}\StringTok{"x"}\NormalTok{,}\DataTypeTok{ylab=}\StringTok{"frequency"}\NormalTok{,}\DataTypeTok{main=}\StringTok{"Random numbers-normal(0,100)"}\NormalTok{)}
\end{Highlighting}
\end{Shaded}

\includegraphics{STAT_757_Assignment1_solutions_files/figure-latex/3.1-1.pdf}

\begin{Shaded}
\begin{Highlighting}[]
\KeywordTok{summary}\NormalTok{(sample1)}
\end{Highlighting}
\end{Shaded}

\begin{verbatim}
##     Min.  1st Qu.   Median     Mean  3rd Qu.     Max. 
## -227.193  -60.885   -5.942    0.291   65.589  258.196
\end{verbatim}

At 0.29, the sample mean is fairly close to the distribution mean of
zero; the min/max values are inline with the standard deviation; and the
shape of the histogram looks like the bell-shaped normal distribution.

\hypertarget{generate-1000-random-variates-from-a-beta-distribution-with-the-parameters-alpha-and-beta-both-equal-to-2.-summarize-and-plot-the-data.-set-a-seed-to-make-it-reproducible.}{%
\subsubsection{\texorpdfstring{3.2 Generate 1000 random variates from a
beta distribution with the parameters \(\alpha\) and \(\beta\) both
equal to 2. Summarize and plot the data. (Set a seed to make it
reproducible).}{3.2 Generate 1000 random variates from a beta distribution with the parameters \textbackslash{}alpha and \textbackslash{}beta both equal to 2. Summarize and plot the data. (Set a seed to make it reproducible).}}\label{generate-1000-random-variates-from-a-beta-distribution-with-the-parameters-alpha-and-beta-both-equal-to-2.-summarize-and-plot-the-data.-set-a-seed-to-make-it-reproducible.}}

\begin{Shaded}
\begin{Highlighting}[]
\KeywordTok{set.seed}\NormalTok{(}\DecValTok{1000}\NormalTok{)}
\NormalTok{sample2<-}\KeywordTok{rbeta}\NormalTok{(}\DecValTok{1000}\NormalTok{,}\DecValTok{2}\NormalTok{,}\DecValTok{2}\NormalTok{)}
\KeywordTok{hist}\NormalTok{(sample2,}\DataTypeTok{xlab=}\StringTok{"x"}\NormalTok{,}\DataTypeTok{ylab=}\StringTok{"frequency"}\NormalTok{,}\DataTypeTok{main=}\StringTok{"Random numbers-beta(2,2)"}\NormalTok{)}
\end{Highlighting}
\end{Shaded}

\includegraphics{STAT_757_Assignment1_solutions_files/figure-latex/3.2-1.pdf}

\begin{Shaded}
\begin{Highlighting}[]
\KeywordTok{summary}\NormalTok{(sample2)}
\end{Highlighting}
\end{Shaded}

\begin{verbatim}
##    Min. 1st Qu.  Median    Mean 3rd Qu.    Max. 
##  0.0211  0.3285  0.5063  0.5003  0.6671  0.9789
\end{verbatim}

With equal parameter values, the distribution should look like a
bell-shaped normal distribution but with a `fatter' bell. This is what
the histogram looks like, though a little skewed to the right. More
samples should smooth this out toward the center. Also, the x values are
between 0 and 1, which is expected for the beta distribution.

\hypertarget{generate-10000-random-variates-from-a-binomial-distribution-with-the-parameters-n10-and-p0.2.-summarize-and-plot-the-data.-set-a-seed-to-make-it-reproducible.}{%
\subsubsection{\texorpdfstring{3.3 Generate 10000 random variates from a
binomial distribution with the parameters \(n=10\) and \(p=0.2\).
Summarize and plot the data. (Set a seed to make it
reproducible).}{3.3 Generate 10000 random variates from a binomial distribution with the parameters n=10 and p=0.2. Summarize and plot the data. (Set a seed to make it reproducible).}}\label{generate-10000-random-variates-from-a-binomial-distribution-with-the-parameters-n10-and-p0.2.-summarize-and-plot-the-data.-set-a-seed-to-make-it-reproducible.}}

\begin{Shaded}
\begin{Highlighting}[]
\KeywordTok{set.seed}\NormalTok{(}\DecValTok{10000}\NormalTok{)}
\NormalTok{sample3<-}\KeywordTok{rbinom}\NormalTok{(}\DecValTok{10000}\NormalTok{,}\DecValTok{10}\NormalTok{,}\FloatTok{0.2}\NormalTok{)}
\KeywordTok{hist}\NormalTok{(sample3,}\DataTypeTok{xlab=}\StringTok{"x"}\NormalTok{,}\DataTypeTok{ylab=}\StringTok{"frequency"}\NormalTok{,}\DataTypeTok{main=}\StringTok{"Random numbers-binomial(10,0.2)"}\NormalTok{)}
\end{Highlighting}
\end{Shaded}

\includegraphics{STAT_757_Assignment1_solutions_files/figure-latex/3.3-1.pdf}

\begin{Shaded}
\begin{Highlighting}[]
\KeywordTok{summary}\NormalTok{(sample3)}
\end{Highlighting}
\end{Shaded}

\begin{verbatim}
##    Min. 1st Qu.  Median    Mean 3rd Qu.    Max. 
##    0.00    1.00    2.00    2.01    3.00    7.00
\end{verbatim}

The sample mean of 2.01 is very close to the expected n and p value - 10
* 0.2 = 2, due to the high sample number (compared to the sample mean in
3.1). And as expected, the distribution is skewed to the lower numbers
(because of the 0.2 parameter) and then tails off sharply at the higher
values.


\end{document}
